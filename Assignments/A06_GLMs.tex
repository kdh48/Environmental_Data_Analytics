\documentclass[]{article}
\usepackage{lmodern}
\usepackage{amssymb,amsmath}
\usepackage{ifxetex,ifluatex}
\usepackage{fixltx2e} % provides \textsubscript
\ifnum 0\ifxetex 1\fi\ifluatex 1\fi=0 % if pdftex
  \usepackage[T1]{fontenc}
  \usepackage[utf8]{inputenc}
\else % if luatex or xelatex
  \ifxetex
    \usepackage{mathspec}
  \else
    \usepackage{fontspec}
  \fi
  \defaultfontfeatures{Ligatures=TeX,Scale=MatchLowercase}
\fi
% use upquote if available, for straight quotes in verbatim environments
\IfFileExists{upquote.sty}{\usepackage{upquote}}{}
% use microtype if available
\IfFileExists{microtype.sty}{%
\usepackage{microtype}
\UseMicrotypeSet[protrusion]{basicmath} % disable protrusion for tt fonts
}{}
\usepackage[margin=2.54cm]{geometry}
\usepackage{hyperref}
\hypersetup{unicode=true,
            pdftitle={Assignment 6: Generalized Linear Models},
            pdfauthor={Kat Horan},
            pdfborder={0 0 0},
            breaklinks=true}
\urlstyle{same}  % don't use monospace font for urls
\usepackage{color}
\usepackage{fancyvrb}
\newcommand{\VerbBar}{|}
\newcommand{\VERB}{\Verb[commandchars=\\\{\}]}
\DefineVerbatimEnvironment{Highlighting}{Verbatim}{commandchars=\\\{\}}
% Add ',fontsize=\small' for more characters per line
\usepackage{framed}
\definecolor{shadecolor}{RGB}{248,248,248}
\newenvironment{Shaded}{\begin{snugshade}}{\end{snugshade}}
\newcommand{\KeywordTok}[1]{\textcolor[rgb]{0.13,0.29,0.53}{\textbf{#1}}}
\newcommand{\DataTypeTok}[1]{\textcolor[rgb]{0.13,0.29,0.53}{#1}}
\newcommand{\DecValTok}[1]{\textcolor[rgb]{0.00,0.00,0.81}{#1}}
\newcommand{\BaseNTok}[1]{\textcolor[rgb]{0.00,0.00,0.81}{#1}}
\newcommand{\FloatTok}[1]{\textcolor[rgb]{0.00,0.00,0.81}{#1}}
\newcommand{\ConstantTok}[1]{\textcolor[rgb]{0.00,0.00,0.00}{#1}}
\newcommand{\CharTok}[1]{\textcolor[rgb]{0.31,0.60,0.02}{#1}}
\newcommand{\SpecialCharTok}[1]{\textcolor[rgb]{0.00,0.00,0.00}{#1}}
\newcommand{\StringTok}[1]{\textcolor[rgb]{0.31,0.60,0.02}{#1}}
\newcommand{\VerbatimStringTok}[1]{\textcolor[rgb]{0.31,0.60,0.02}{#1}}
\newcommand{\SpecialStringTok}[1]{\textcolor[rgb]{0.31,0.60,0.02}{#1}}
\newcommand{\ImportTok}[1]{#1}
\newcommand{\CommentTok}[1]{\textcolor[rgb]{0.56,0.35,0.01}{\textit{#1}}}
\newcommand{\DocumentationTok}[1]{\textcolor[rgb]{0.56,0.35,0.01}{\textbf{\textit{#1}}}}
\newcommand{\AnnotationTok}[1]{\textcolor[rgb]{0.56,0.35,0.01}{\textbf{\textit{#1}}}}
\newcommand{\CommentVarTok}[1]{\textcolor[rgb]{0.56,0.35,0.01}{\textbf{\textit{#1}}}}
\newcommand{\OtherTok}[1]{\textcolor[rgb]{0.56,0.35,0.01}{#1}}
\newcommand{\FunctionTok}[1]{\textcolor[rgb]{0.00,0.00,0.00}{#1}}
\newcommand{\VariableTok}[1]{\textcolor[rgb]{0.00,0.00,0.00}{#1}}
\newcommand{\ControlFlowTok}[1]{\textcolor[rgb]{0.13,0.29,0.53}{\textbf{#1}}}
\newcommand{\OperatorTok}[1]{\textcolor[rgb]{0.81,0.36,0.00}{\textbf{#1}}}
\newcommand{\BuiltInTok}[1]{#1}
\newcommand{\ExtensionTok}[1]{#1}
\newcommand{\PreprocessorTok}[1]{\textcolor[rgb]{0.56,0.35,0.01}{\textit{#1}}}
\newcommand{\AttributeTok}[1]{\textcolor[rgb]{0.77,0.63,0.00}{#1}}
\newcommand{\RegionMarkerTok}[1]{#1}
\newcommand{\InformationTok}[1]{\textcolor[rgb]{0.56,0.35,0.01}{\textbf{\textit{#1}}}}
\newcommand{\WarningTok}[1]{\textcolor[rgb]{0.56,0.35,0.01}{\textbf{\textit{#1}}}}
\newcommand{\AlertTok}[1]{\textcolor[rgb]{0.94,0.16,0.16}{#1}}
\newcommand{\ErrorTok}[1]{\textcolor[rgb]{0.64,0.00,0.00}{\textbf{#1}}}
\newcommand{\NormalTok}[1]{#1}
\usepackage{graphicx,grffile}
\makeatletter
\def\maxwidth{\ifdim\Gin@nat@width>\linewidth\linewidth\else\Gin@nat@width\fi}
\def\maxheight{\ifdim\Gin@nat@height>\textheight\textheight\else\Gin@nat@height\fi}
\makeatother
% Scale images if necessary, so that they will not overflow the page
% margins by default, and it is still possible to overwrite the defaults
% using explicit options in \includegraphics[width, height, ...]{}
\setkeys{Gin}{width=\maxwidth,height=\maxheight,keepaspectratio}
\IfFileExists{parskip.sty}{%
\usepackage{parskip}
}{% else
\setlength{\parindent}{0pt}
\setlength{\parskip}{6pt plus 2pt minus 1pt}
}
\setlength{\emergencystretch}{3em}  % prevent overfull lines
\providecommand{\tightlist}{%
  \setlength{\itemsep}{0pt}\setlength{\parskip}{0pt}}
\setcounter{secnumdepth}{0}
% Redefines (sub)paragraphs to behave more like sections
\ifx\paragraph\undefined\else
\let\oldparagraph\paragraph
\renewcommand{\paragraph}[1]{\oldparagraph{#1}\mbox{}}
\fi
\ifx\subparagraph\undefined\else
\let\oldsubparagraph\subparagraph
\renewcommand{\subparagraph}[1]{\oldsubparagraph{#1}\mbox{}}
\fi

%%% Use protect on footnotes to avoid problems with footnotes in titles
\let\rmarkdownfootnote\footnote%
\def\footnote{\protect\rmarkdownfootnote}

%%% Change title format to be more compact
\usepackage{titling}

% Create subtitle command for use in maketitle
\newcommand{\subtitle}[1]{
  \posttitle{
    \begin{center}\large#1\end{center}
    }
}

\setlength{\droptitle}{-2em}

  \title{Assignment 6: Generalized Linear Models}
    \pretitle{\vspace{\droptitle}\centering\huge}
  \posttitle{\par}
    \author{Kat Horan}
    \preauthor{\centering\large\emph}
  \postauthor{\par}
    \date{}
    \predate{}\postdate{}
  

\begin{document}
\maketitle

\subsection{OVERVIEW}\label{overview}

This exercise accompanies the lessons in Environmental Data Analytics
(ENV872L) on generalized linear models.

\subsection{Directions}\label{directions}

\begin{enumerate}
\def\labelenumi{\arabic{enumi}.}
\tightlist
\item
  Change ``Student Name'' on line 3 (above) with your name.
\item
  Use the lesson as a guide. It contains code that can be modified to
  complete the assignment.
\item
  Work through the steps, \textbf{creating code and output} that fulfill
  each instruction.
\item
  Be sure to \textbf{answer the questions} in this assignment document.
  Space for your answers is provided in this document and is indicated
  by the ``\textgreater{}'' character. If you need a second paragraph be
  sure to start the first line with ``\textgreater{}''. You should
  notice that the answer is highlighted in green by RStudio.
\item
  When you have completed the assignment, \textbf{Knit} the text and
  code into a single PDF file. You will need to have the correct
  software installed to do this (see Software Installation Guide) Press
  the \texttt{Knit} button in the RStudio scripting panel. This will
  save the PDF output in your Assignments folder.
\item
  After Knitting, please submit the completed exercise (PDF file) to the
  dropbox in Sakai. Please add your last name into the file name (e.g.,
  ``Salk\_A06\_GLMs.pdf'') prior to submission.
\end{enumerate}

The completed exercise is due on Tuesday, 26 February, 2019 before class
begins.

\subsection{Set up your session}\label{set-up-your-session}

\begin{enumerate}
\def\labelenumi{\arabic{enumi}.}
\item
  Set up your session. Upload the EPA Ecotox dataset for Neonicotinoids
  and the NTL-LTER raw data file for chemistry/physics.
\item
  Build a ggplot theme and set it as your default theme.
\end{enumerate}

\begin{Shaded}
\begin{Highlighting}[]
\CommentTok{#1 Set up session}

\CommentTok{# Set working directory:}
\CommentTok{# setwd("/Users/kathleenhoran/Desktop/Duke/Spring 2019/Env. Data Analytics/Env_Data_Analytics")}

\CommentTok{# Load packages:}
\KeywordTok{library}\NormalTok{(ggplot2)}
\KeywordTok{library}\NormalTok{(viridis)}
\end{Highlighting}
\end{Shaded}

\begin{verbatim}
## Loading required package: viridisLite
\end{verbatim}

\begin{Shaded}
\begin{Highlighting}[]
\KeywordTok{library}\NormalTok{(RColorBrewer)}
\KeywordTok{library}\NormalTok{(tidyverse)}
\end{Highlighting}
\end{Shaded}

\begin{verbatim}
## -- Attaching packages ---------------------------------------------- tidyverse 1.2.1 --
\end{verbatim}

\begin{verbatim}
## v tibble  2.0.1       v purrr   0.3.0  
## v tidyr   0.8.2       v dplyr   0.8.0.1
## v readr   1.3.1       v stringr 1.4.0  
## v tibble  2.0.1       v forcats 0.4.0
\end{verbatim}

\begin{verbatim}
## -- Conflicts ------------------------------------------------- tidyverse_conflicts() --
## x dplyr::filter() masks stats::filter()
## x dplyr::lag()    masks stats::lag()
\end{verbatim}

\begin{Shaded}
\begin{Highlighting}[]
\KeywordTok{library}\NormalTok{(gridExtra)}
\end{Highlighting}
\end{Shaded}

\begin{verbatim}
## 
## Attaching package: 'gridExtra'
\end{verbatim}

\begin{verbatim}
## The following object is masked from 'package:dplyr':
## 
##     combine
\end{verbatim}

\begin{Shaded}
\begin{Highlighting}[]
\KeywordTok{library}\NormalTok{(dunn.test)}

\CommentTok{# Load datasets:}
\CommentTok{# NTL-LTER for chemistry/physics raw data}
\NormalTok{NTL.lake.chem.phys.raw <-}
\StringTok{  }\KeywordTok{read.csv}\NormalTok{(}\StringTok{"./Data/Raw/NTL-LTER_Lake_ChemistryPhysics_Raw.csv"}\NormalTok{)}

\CommentTok{# Ecotox for Neonicotinoids raw data}
\NormalTok{Ecotox.neo.mort.raw <-}\StringTok{ }\KeywordTok{read.csv}\NormalTok{(}\StringTok{"./Data/Raw/ECOTOX_Neonicotinoids_Mortality_raw.csv"}\NormalTok{)}

\CommentTok{#2 Set ggplot theme}
\NormalTok{mytheme <-}\StringTok{ }\KeywordTok{theme_classic}\NormalTok{(}\DataTypeTok{base_size =} \DecValTok{14}\NormalTok{) }\OperatorTok{+}
\StringTok{  }\KeywordTok{theme}\NormalTok{(}\DataTypeTok{axis.text =} \KeywordTok{element_text}\NormalTok{(}\DataTypeTok{color =} \StringTok{"black"}\NormalTok{),}
        \DataTypeTok{legend.position =} \StringTok{"top"}\NormalTok{)}
\KeywordTok{theme_set}\NormalTok{(mytheme)}
\end{Highlighting}
\end{Shaded}

\subsection{Neonicotinoids test}\label{neonicotinoids-test}

Research question: Were studies on various neonicotinoid chemicals
conducted in different years?

\begin{enumerate}
\def\labelenumi{\arabic{enumi}.}
\setcounter{enumi}{2}
\item
  Generate a line of code to determine how many different chemicals are
  listed in the Chemical.Name column.
\item
  Are the publication years associated with each chemical
  well-approximated by a normal distribution? Run the appropriate test
  and also generate a frequency polygon to illustrate the distribution
  of counts for each year, divided by chemical name. Bonus points if you
  can generate the results of your test from a pipe function. No need to
  make this graph pretty.
\item
  Is there equal variance among the publication years for each chemical?
  Hint: var.test is not the correct function.
\end{enumerate}

\begin{Shaded}
\begin{Highlighting}[]
\CommentTok{#3 Check different chemicals listed in Chemical.Name column }
\KeywordTok{levels}\NormalTok{(Ecotox.neo.mort.raw}\OperatorTok{$}\NormalTok{Chemical.Name)}
\end{Highlighting}
\end{Shaded}

\begin{verbatim}
## [1] "Acetamiprid"  "Clothianidin" "Dinotefuran"  "Imidacloprid"
## [5] "Imidaclothiz" "Nitenpyram"   "Nithiazine"   "Thiacloprid" 
## [9] "Thiamethoxam"
\end{verbatim}

\begin{Shaded}
\begin{Highlighting}[]
\KeywordTok{summary.factor}\NormalTok{(Ecotox.neo.mort.raw}\OperatorTok{$}\NormalTok{Chemical.Name)}
\end{Highlighting}
\end{Shaded}

\begin{verbatim}
##  Acetamiprid Clothianidin  Dinotefuran Imidacloprid Imidaclothiz 
##          136           74           59          695            9 
##   Nitenpyram   Nithiazine  Thiacloprid Thiamethoxam 
##           21           22          106          161
\end{verbatim}

\begin{Shaded}
\begin{Highlighting}[]
\CommentTok{#4}
\CommentTok{# Shapiro test for normality for publication years associated with each of the nine chemicals}
\KeywordTok{shapiro.test}\NormalTok{(Ecotox.neo.mort.raw}\OperatorTok{$}\NormalTok{Pub..Year[Ecotox.neo.mort.raw}\OperatorTok{$}\NormalTok{Chemical.Name }\OperatorTok{==}\StringTok{ "Acetamiprid"}\NormalTok{])}
\end{Highlighting}
\end{Shaded}

\begin{verbatim}
## 
##  Shapiro-Wilk normality test
## 
## data:  Ecotox.neo.mort.raw$Pub..Year[Ecotox.neo.mort.raw$Chemical.Name ==     "Acetamiprid"]
## W = 0.90191, p-value = 5.706e-08
\end{verbatim}

\begin{Shaded}
\begin{Highlighting}[]
\KeywordTok{shapiro.test}\NormalTok{(Ecotox.neo.mort.raw}\OperatorTok{$}\NormalTok{Pub..Year[Ecotox.neo.mort.raw}\OperatorTok{$}\NormalTok{Chemical.Name }\OperatorTok{==}\StringTok{ "Clothianidin"}\NormalTok{])}
\end{Highlighting}
\end{Shaded}

\begin{verbatim}
## 
##  Shapiro-Wilk normality test
## 
## data:  Ecotox.neo.mort.raw$Pub..Year[Ecotox.neo.mort.raw$Chemical.Name ==     "Clothianidin"]
## W = 0.69577, p-value = 4.287e-11
\end{verbatim}

\begin{Shaded}
\begin{Highlighting}[]
\KeywordTok{shapiro.test}\NormalTok{(Ecotox.neo.mort.raw}\OperatorTok{$}\NormalTok{Pub..Year[Ecotox.neo.mort.raw}\OperatorTok{$}\NormalTok{Chemical.Name }\OperatorTok{==}\StringTok{ "Dinotefuran"}\NormalTok{])}
\end{Highlighting}
\end{Shaded}

\begin{verbatim}
## 
##  Shapiro-Wilk normality test
## 
## data:  Ecotox.neo.mort.raw$Pub..Year[Ecotox.neo.mort.raw$Chemical.Name ==     "Dinotefuran"]
## W = 0.82848, p-value = 8.83e-07
\end{verbatim}

\begin{Shaded}
\begin{Highlighting}[]
\KeywordTok{shapiro.test}\NormalTok{(Ecotox.neo.mort.raw}\OperatorTok{$}\NormalTok{Pub..Year[Ecotox.neo.mort.raw}\OperatorTok{$}\NormalTok{Chemical.Name }\OperatorTok{==}\StringTok{ "Imidacloprid"}\NormalTok{])}
\end{Highlighting}
\end{Shaded}

\begin{verbatim}
## 
##  Shapiro-Wilk normality test
## 
## data:  Ecotox.neo.mort.raw$Pub..Year[Ecotox.neo.mort.raw$Chemical.Name ==     "Imidacloprid"]
## W = 0.88178, p-value < 2.2e-16
\end{verbatim}

\begin{Shaded}
\begin{Highlighting}[]
\KeywordTok{shapiro.test}\NormalTok{(Ecotox.neo.mort.raw}\OperatorTok{$}\NormalTok{Pub..Year[Ecotox.neo.mort.raw}\OperatorTok{$}\NormalTok{Chemical.Name }\OperatorTok{==}\StringTok{ "Imidaclothiz"}\NormalTok{])}
\end{Highlighting}
\end{Shaded}

\begin{verbatim}
## 
##  Shapiro-Wilk normality test
## 
## data:  Ecotox.neo.mort.raw$Pub..Year[Ecotox.neo.mort.raw$Chemical.Name ==     "Imidaclothiz"]
## W = 0.68429, p-value = 0.00093
\end{verbatim}

\begin{Shaded}
\begin{Highlighting}[]
\KeywordTok{shapiro.test}\NormalTok{(Ecotox.neo.mort.raw}\OperatorTok{$}\NormalTok{Pub..Year[Ecotox.neo.mort.raw}\OperatorTok{$}\NormalTok{Chemical.Name }\OperatorTok{==}\StringTok{ "Nitenpyram"}\NormalTok{])}
\end{Highlighting}
\end{Shaded}

\begin{verbatim}
## 
##  Shapiro-Wilk normality test
## 
## data:  Ecotox.neo.mort.raw$Pub..Year[Ecotox.neo.mort.raw$Chemical.Name ==     "Nitenpyram"]
## W = 0.79592, p-value = 0.0005686
\end{verbatim}

\begin{Shaded}
\begin{Highlighting}[]
\KeywordTok{shapiro.test}\NormalTok{(Ecotox.neo.mort.raw}\OperatorTok{$}\NormalTok{Pub..Year[Ecotox.neo.mort.raw}\OperatorTok{$}\NormalTok{Chemical.Name }\OperatorTok{==}\StringTok{ "Nithiazine"}\NormalTok{])}
\end{Highlighting}
\end{Shaded}

\begin{verbatim}
## 
##  Shapiro-Wilk normality test
## 
## data:  Ecotox.neo.mort.raw$Pub..Year[Ecotox.neo.mort.raw$Chemical.Name ==     "Nithiazine"]
## W = 0.75938, p-value = 0.0001235
\end{verbatim}

\begin{Shaded}
\begin{Highlighting}[]
\KeywordTok{shapiro.test}\NormalTok{(Ecotox.neo.mort.raw}\OperatorTok{$}\NormalTok{Pub..Year[Ecotox.neo.mort.raw}\OperatorTok{$}\NormalTok{Chemical.Name }\OperatorTok{==}\StringTok{ "Thiacloprid"}\NormalTok{])}
\end{Highlighting}
\end{Shaded}

\begin{verbatim}
## 
##  Shapiro-Wilk normality test
## 
## data:  Ecotox.neo.mort.raw$Pub..Year[Ecotox.neo.mort.raw$Chemical.Name ==     "Thiacloprid"]
## W = 0.7669, p-value = 1.118e-11
\end{verbatim}

\begin{Shaded}
\begin{Highlighting}[]
\KeywordTok{shapiro.test}\NormalTok{(Ecotox.neo.mort.raw}\OperatorTok{$}\NormalTok{Pub..Year[Ecotox.neo.mort.raw}\OperatorTok{$}\NormalTok{Chemical.Name }\OperatorTok{==}\StringTok{ "Thiamethoxam"}\NormalTok{])}
\end{Highlighting}
\end{Shaded}

\begin{verbatim}
## 
##  Shapiro-Wilk normality test
## 
## data:  Ecotox.neo.mort.raw$Pub..Year[Ecotox.neo.mort.raw$Chemical.Name ==     "Thiamethoxam"]
## W = 0.7071, p-value < 2.2e-16
\end{verbatim}

\begin{Shaded}
\begin{Highlighting}[]
\CommentTok{# The publication years are not well-approximated by a normal distribution. }
\CommentTok{# Since the p-values generated by the Shapiro-Wilk tests for pub years associated}
\CommentTok{# with each chemical are all smaller than .05, we reject the null hypothesis of}
\CommentTok{# normality.}

\CommentTok{# Frequency polygon}
\KeywordTok{ggplot}\NormalTok{(Ecotox.neo.mort.raw, }\KeywordTok{aes}\NormalTok{(}\DataTypeTok{x =}\NormalTok{ Pub..Year, }\DataTypeTok{color =}\NormalTok{ Chemical.Name)) }\OperatorTok{+}
\StringTok{  }\KeywordTok{geom_freqpoly}\NormalTok{(}\DataTypeTok{stat =} \StringTok{"count"}\NormalTok{) }\OperatorTok{+}
\StringTok{  }\KeywordTok{theme}\NormalTok{(}\DataTypeTok{legend.position =} \StringTok{"right"}\NormalTok{)}
\end{Highlighting}
\end{Shaded}

\begin{center}\includegraphics{A06_GLMs_files/figure-latex/unnamed-chunk-2-1} \end{center}

\begin{Shaded}
\begin{Highlighting}[]
\CommentTok{#5 Test for equal variance}
\KeywordTok{bartlett.test}\NormalTok{(Ecotox.neo.mort.raw}\OperatorTok{$}\NormalTok{Pub..Year }\OperatorTok{~}\StringTok{ }\NormalTok{Ecotox.neo.mort.raw}\OperatorTok{$}\NormalTok{Chemical.Name)}
\end{Highlighting}
\end{Shaded}

\begin{verbatim}
## 
##  Bartlett test of homogeneity of variances
## 
## data:  Ecotox.neo.mort.raw$Pub..Year by Ecotox.neo.mort.raw$Chemical.Name
## Bartlett's K-squared = 139.59, df = 8, p-value < 2.2e-16
\end{verbatim}

\begin{Shaded}
\begin{Highlighting}[]
\CommentTok{# There is not equal variance among the publication years for each chemical, as}
\CommentTok{# indicated by the low p-value from the Bartlett test. We accept the Bartlett }
\CommentTok{# test's alternative hypothesis that the variances are not equal}
\end{Highlighting}
\end{Shaded}

\begin{enumerate}
\def\labelenumi{\arabic{enumi}.}
\setcounter{enumi}{5}
\tightlist
\item
  Based on your results, which test would you choose to run to answer
  your research question?
\end{enumerate}

\begin{quote}
ANSWER: I would select the Kruskal Wallace test because it is not
normally distributed.
\end{quote}

\begin{enumerate}
\def\labelenumi{\arabic{enumi}.}
\setcounter{enumi}{6}
\item
  Run this test below.
\item
  Generate a boxplot representing the range of publication years for
  each chemical. Adjust your graph to make it pretty.
\end{enumerate}

\begin{Shaded}
\begin{Highlighting}[]
\CommentTok{#7 Kruskal Wallace test}
\NormalTok{chem.pub.kw <-}\StringTok{ }\KeywordTok{kruskal.test}\NormalTok{(Ecotox.neo.mort.raw}\OperatorTok{$}\NormalTok{Pub..Year }\OperatorTok{~}\StringTok{ }\NormalTok{Ecotox.neo.mort.raw}\OperatorTok{$}\NormalTok{Chemical.Name)}
\NormalTok{chem.pub.kw}
\end{Highlighting}
\end{Shaded}

\begin{verbatim}
## 
##  Kruskal-Wallis rank sum test
## 
## data:  Ecotox.neo.mort.raw$Pub..Year by Ecotox.neo.mort.raw$Chemical.Name
## Kruskal-Wallis chi-squared = 134.15, df = 8, p-value < 2.2e-16
\end{verbatim}

\begin{Shaded}
\begin{Highlighting}[]
\KeywordTok{dunn.test}\NormalTok{(Ecotox.neo.mort.raw}\OperatorTok{$}\NormalTok{Pub..Year, Ecotox.neo.mort.raw}\OperatorTok{$}\NormalTok{Chemical.Name, }\DataTypeTok{kw =}\NormalTok{ T, }
           \DataTypeTok{table =}\NormalTok{ F, }\DataTypeTok{list =}\NormalTok{ T, }\DataTypeTok{method =} \StringTok{"holm"}\NormalTok{, }\DataTypeTok{altp =}\NormalTok{ T)}
\end{Highlighting}
\end{Shaded}

\begin{verbatim}
##   Kruskal-Wallis rank sum test
## 
## data: x and group
## Kruskal-Wallis chi-squared = 134.1455, df = 8, p-value = 0
## 
## 
##                            Comparison of x by group                            
##                                     (Holm)                                     
## 
## List of pairwise comparisons: Z statistic (adjusted p-value)
## -------------------------------------------------
## Acetamiprid - Clothianidin  : -3.038807 (0.0404)*
## Acetamiprid - Dinotefuran   :  2.117208 (0.4109)
## Clothianidin - Dinotefuran  :  4.406076 (0.0002)*
## Acetamiprid - Imidacloprid  : -4.020498 (0.0013)*
## Clothianidin - Imidacloprid :  0.506889 (1.0000)
## Dinotefuran - Imidacloprid  : -5.214028 (0.0000)*
## Acetamiprid - Imidaclothiz  : -1.805293 (0.7813)
## Clothianidin - Imidaclothiz : -0.516664 (1.0000)
## Dinotefuran - Imidaclothiz  : -2.658649 (0.1177)
## Imidacloprid - Imidaclothiz : -0.728428 (1.0000)
## Acetamiprid - Nitenpyram    : -4.501863 (0.0002)*
## Clothianidin - Nitenpyram   : -2.493626 (0.1770)
## Dinotefuran - Nitenpyram    : -5.452779 (0.0000)*
## Imidacloprid - Nitenpyram   : -3.063483 (0.0394)*
## Imidaclothiz - Nitenpyram   : -1.089720 (1.0000)
## Acetamiprid - Nithiazine    :  5.642529 (0.0000)*
## Clothianidin - Nithiazine   :  7.147325 (0.0000)*
## Dinotefuran - Nithiazine    :  3.869350 (0.0023)*
## Imidacloprid - Nithiazine   :  7.728634 (0.0000)*
## Imidaclothiz - Nithiazine   :  4.847313 (0.0000)*
## Nitenpyram - Nithiazine     :  7.709981 (0.0000)*
## Acetamiprid - Thiacloprid   : -3.222561 (0.0241)*
## Clothianidin - Thiacloprid  :  0.141491 (0.8875)
## Dinotefuran - Thiacloprid   : -4.602529 (0.0001)*
## Imidacloprid - Thiacloprid  : -0.388871 (1.0000)
## Imidaclothiz - Thiacloprid  :  0.587068 (1.0000)
## Nitenpyram - Thiacloprid    :  2.670974 (0.1210)
## Nithiazine - Thiacloprid    : -7.316688 (0.0000)*
## Acetamiprid - Thiamethoxam  : -5.889886 (0.0000)*
## Clothianidin - Thiamethoxam : -1.758725 (0.7862)
## Dinotefuran - Thiamethoxam  : -6.676212 (0.0000)*
## Imidacloprid - Thiamethoxam : -3.532703 (0.0082)*
## Imidaclothiz - Thiamethoxam : -0.188627 (1.0000)
## Nitenpyram - Thiamethoxam   :  1.592776 (1.0000)
## Nithiazine - Thiamethoxam   : -8.722412 (0.0000)*
## Thiacloprid - Thiamethoxam  : -2.146115 (0.4142)
## 
## alpha = 0.05
## Reject Ho if p <= alpha
\end{verbatim}

\begin{Shaded}
\begin{Highlighting}[]
\CommentTok{#8 Boxplot with range of publication years for each chemical}
\NormalTok{chem.pub.boxplot <-}
\StringTok{  }\KeywordTok{ggplot}\NormalTok{(Ecotox.neo.mort.raw, }\KeywordTok{aes}\NormalTok{(}\DataTypeTok{x =}\NormalTok{ Chemical.Name, }\DataTypeTok{y =}\NormalTok{ Pub..Year, }\DataTypeTok{color =}\NormalTok{ Chemical.Name)) }\OperatorTok{+}
\StringTok{  }\KeywordTok{geom_boxplot}\NormalTok{() }\OperatorTok{+}
\StringTok{  }\KeywordTok{labs}\NormalTok{(}\DataTypeTok{x =} \StringTok{"Chemical Name"}\NormalTok{, }\DataTypeTok{y =} \StringTok{"Publication Year"}\NormalTok{) }\OperatorTok{+}
\StringTok{  }\KeywordTok{scale_y_continuous}\NormalTok{(}\DataTypeTok{expand =} \KeywordTok{c}\NormalTok{(}\DecValTok{0}\NormalTok{, }\DecValTok{0}\NormalTok{)) }\OperatorTok{+}
\StringTok{  }\KeywordTok{scale_color_viridis}\NormalTok{(}\DataTypeTok{discrete =} \OtherTok{TRUE}\NormalTok{) }\OperatorTok{+}
\StringTok{  }\KeywordTok{theme}\NormalTok{(}\DataTypeTok{legend.position =} \StringTok{"none"}\NormalTok{, }\DataTypeTok{axis.text.x =} \KeywordTok{element_text}\NormalTok{(}\DataTypeTok{angle =} \DecValTok{45}\NormalTok{,  }\DataTypeTok{hjust =} \DecValTok{1}\NormalTok{))}
\KeywordTok{print}\NormalTok{(chem.pub.boxplot)}
\end{Highlighting}
\end{Shaded}

\begin{center}\includegraphics{A06_GLMs_files/figure-latex/unnamed-chunk-3-1} \end{center}

\begin{enumerate}
\def\labelenumi{\arabic{enumi}.}
\setcounter{enumi}{8}
\tightlist
\item
  How would you summarize the conclusion of your analysis? Include a
  sentence summarizing your findings and include the results of your
  test in parentheses at the end of the sentence.
\end{enumerate}

\begin{quote}
ANSWER: The p-value smaller than 0.05 from the Kruskal-Wallis test
indicates that there is a significant difference between groups.
(Kruskal-Wallis chi-squared = 134.1455, df = 8, p-value = 0).
\end{quote}

\begin{quote}
Since it does not indicate which groups, I ran a Dunn test to see which
pairs had the significant difference. We see that the following pairs
are significantly different:
\end{quote}

\begin{quote}
Acetamiprid - Clothianidin : -3.038807 (0.0404)\emph{ Clothianidin -
Dinotefuran : 4.406076 (0.0002)} Acetamiprid - Imidacloprid : -4.020498
(0.0013)\emph{ Dinotefuran - Imidacloprid : -5.214028 (0.0000)}
Acetamiprid - Nitenpyram : -4.501863 (0.0002)\emph{ Dinotefuran -
Nitenpyram : -5.452779 (0.0000)} Imidacloprid - Nitenpyram : -3.063483
(0.0394)\emph{ Acetamiprid - Nithiazine : 5.642529 (0.0000)}
Clothianidin - Nithiazine : 7.147325 (0.0000)\emph{ Dinotefuran -
Nithiazine : 3.869350 (0.0023)} Imidacloprid - Nithiazine : 7.728634
(0.0000)\emph{ Imidaclothiz - Nithiazine : 4.847313 (0.0000)} Nitenpyram
- Nithiazine : 7.709981 (0.0000)\emph{ Acetamiprid - Thiacloprid :
-3.222561 (0.0241)} Dinotefuran - Thiacloprid : -4.602529 (0.0001)\emph{
Nithiazine - Thiacloprid : -7.316688 (0.0000)} Acetamiprid -
Thiamethoxam : -5.889886 (0.0000)\emph{ Dinotefuran - Thiamethoxam :
-6.676212 (0.0000)} Imidacloprid - Thiamethoxam : -3.532703
(0.0082)\emph{ Nithiazine - Thiamethoxam : -8.722412 (0.0000)}
\end{quote}

\subsection{NTL-LTER test}\label{ntl-lter-test}

Research question: What is the best set of predictors for lake
temperatures in July across the monitoring period at the North Temperate
Lakes LTER?

\begin{enumerate}
\def\labelenumi{\arabic{enumi}.}
\setcounter{enumi}{10}
\tightlist
\item
  Wrangle your NTL-LTER dataset with a pipe function so that it contains
  only the following criteria:
\end{enumerate}

\begin{itemize}
\tightlist
\item
  Only dates in July (hint: use the daynum column). No need to consider
  leap years.
\item
  Only the columns: lakename, year4, daynum, depth, temperature\_C
\item
  Only complete cases (i.e., remove NAs)
\end{itemize}

\begin{enumerate}
\def\labelenumi{\arabic{enumi}.}
\setcounter{enumi}{11}
\tightlist
\item
  Run an AIC to determine what set of explanatory variables (year4,
  daynum, depth) is best suited to predict temperature. Run a multiple
  regression on the recommended set of variables.
\end{enumerate}

\begin{Shaded}
\begin{Highlighting}[]
\CommentTok{#11 Wrangling data set}
\NormalTok{NTL.lake.chem.phys.processed <-}\StringTok{ }
\StringTok{  }\NormalTok{NTL.lake.chem.phys.raw }\OperatorTok
\StringTok{  }\KeywordTok{filter}\NormalTok{(daynum }\OperatorTok{>=}\StringTok{ }\DecValTok{182} \OperatorTok{&}\StringTok{ }\NormalTok{daynum }\OperatorTok{<=}\StringTok{ }\DecValTok{212}\NormalTok{) }\OperatorTok
\StringTok{  }\KeywordTok{select}\NormalTok{(lakename, year4, daynum, depth, temperature_C) }\OperatorTok
\StringTok{  }\KeywordTok{filter}\NormalTok{(}\OperatorTok{!}\KeywordTok{is.na}\NormalTok{(temperature_C))}

\CommentTok{#12 }
\CommentTok{# AIC tests with all explanatory variables}
\NormalTok{NTL.lake.chem.phys.AIC <-}\StringTok{ }
\StringTok{  }\KeywordTok{lm}\NormalTok{(}\DataTypeTok{data =}\NormalTok{ NTL.lake.chem.phys.processed, temperature_C }\OperatorTok{~}\StringTok{ }\NormalTok{year4 }\OperatorTok{+}\StringTok{ }\NormalTok{daynum }\OperatorTok{+}\StringTok{ }\NormalTok{depth)}
\KeywordTok{step}\NormalTok{(NTL.lake.chem.phys.AIC)}
\end{Highlighting}
\end{Shaded}

\begin{verbatim}
## Start:  AIC=26016.31
## temperature_C ~ year4 + daynum + depth
## 
##          Df Sum of Sq    RSS   AIC
## <none>                141118 26016
## - year4   1        80 141198 26020
## - daynum  1      1333 142450 26106
## - depth   1    403925 545042 39151
\end{verbatim}

\begin{verbatim}
## 
## Call:
## lm(formula = temperature_C ~ year4 + daynum + depth, data = NTL.lake.chem.phys.processed)
## 
## Coefficients:
## (Intercept)        year4       daynum        depth  
##    -6.45556      0.01013      0.04134     -1.94726
\end{verbatim}

\begin{Shaded}
\begin{Highlighting}[]
\CommentTok{# Final multiple regression}
\NormalTok{NTL.lake.chem.phys.model <-}\StringTok{ }
\StringTok{  }\KeywordTok{lm}\NormalTok{(}\DataTypeTok{data =}\NormalTok{ NTL.lake.chem.phys.processed, temperature_C }\OperatorTok{~}\StringTok{ }\NormalTok{year4 }\OperatorTok{+}\StringTok{ }\NormalTok{daynum }\OperatorTok{+}\StringTok{ }\NormalTok{depth)}
\KeywordTok{summary}\NormalTok{(NTL.lake.chem.phys.model)}
\end{Highlighting}
\end{Shaded}

\begin{verbatim}
## 
## Call:
## lm(formula = temperature_C ~ year4 + daynum + depth, data = NTL.lake.chem.phys.processed)
## 
## Residuals:
##     Min      1Q  Median      3Q     Max 
## -9.6517 -2.9937  0.0855  2.9692 13.6171 
## 
## Coefficients:
##              Estimate Std. Error  t value Pr(>|t|)    
## (Intercept) -6.455560   8.638808   -0.747   0.4549    
## year4        0.010131   0.004303    2.354   0.0186 *  
## daynum       0.041336   0.004315    9.580   <2e-16 ***
## depth       -1.947264   0.011676 -166.782   <2e-16 ***
## ---
## Signif. codes:  0 '***' 0.001 '**' 0.01 '*' 0.05 '.' 0.1 ' ' 1
## 
## Residual standard error: 3.811 on 9718 degrees of freedom
## Multiple R-squared:  0.7417, Adjusted R-squared:  0.7417 
## F-statistic:  9303 on 3 and 9718 DF,  p-value: < 2.2e-16
\end{verbatim}

\begin{enumerate}
\def\labelenumi{\arabic{enumi}.}
\setcounter{enumi}{12}
\tightlist
\item
  What is the final linear equation to predict temperature from your
  multiple regression? How much of the observed variance does this model
  explain?
\end{enumerate}

\begin{quote}
ANSWER: The final linear equation is also the original linear model. The
AIC test found that all three explanatory variables were significant and
did not find that it needed to be reduced. 74.17\% of the variance is
explained in this mode.
\end{quote}

\begin{enumerate}
\def\labelenumi{\arabic{enumi}.}
\setcounter{enumi}{13}
\tightlist
\item
  Run an interaction effects ANCOVA to predict temperature based on
  depth and lakename from the same wrangled dataset.
\end{enumerate}

\begin{Shaded}
\begin{Highlighting}[]
\CommentTok{#14 Interaction effects ANCOVA}
\NormalTok{NTL.lake.chem.phys.ancova.int <-}\StringTok{ }
\StringTok{  }\KeywordTok{lm}\NormalTok{(}\DataTypeTok{data =}\NormalTok{ NTL.lake.chem.phys.processed, temperature_C }\OperatorTok{~}\StringTok{ }\NormalTok{lakename }\OperatorTok{*}\StringTok{ }\NormalTok{depth)}
\KeywordTok{summary}\NormalTok{(NTL.lake.chem.phys.ancova.int)}
\end{Highlighting}
\end{Shaded}

\begin{verbatim}
## 
## Call:
## lm(formula = temperature_C ~ lakename * depth, data = NTL.lake.chem.phys.processed)
## 
## Residuals:
##     Min      1Q  Median      3Q     Max 
## -7.6455 -2.9133 -0.2879  2.7567 16.3606 
## 
## Coefficients:
##                                Estimate Std. Error t value Pr(>|t|)    
## (Intercept)                     22.9455     0.5861  39.147  < 2e-16 ***
## lakenameCrampton Lake            2.2173     0.6804   3.259  0.00112 ** 
## lakenameEast Long Lake          -4.3884     0.6191  -7.089 1.45e-12 ***
## lakenameHummingbird Lake        -2.4126     0.8379  -2.879  0.00399 ** 
## lakenamePaul Lake                0.6105     0.5983   1.020  0.30754    
## lakenamePeter Lake               0.2998     0.5970   0.502  0.61552    
## lakenameTuesday Lake            -2.8932     0.6060  -4.774 1.83e-06 ***
## lakenameWard Lake                2.4180     0.8434   2.867  0.00415 ** 
## lakenameWest Long Lake          -2.4663     0.6168  -3.999 6.42e-05 ***
## depth                           -2.5820     0.2411 -10.711  < 2e-16 ***
## lakenameCrampton Lake:depth      0.8058     0.2465   3.268  0.00109 ** 
## lakenameEast Long Lake:depth     0.9465     0.2433   3.891  0.00010 ***
## lakenameHummingbird Lake:depth  -0.6026     0.2919  -2.064  0.03903 *  
## lakenamePaul Lake:depth          0.4022     0.2421   1.662  0.09664 .  
## lakenamePeter Lake:depth         0.5799     0.2418   2.398  0.01649 *  
## lakenameTuesday Lake:depth       0.6605     0.2426   2.723  0.00648 ** 
## lakenameWard Lake:depth         -0.6930     0.2862  -2.421  0.01548 *  
## lakenameWest Long Lake:depth     0.8154     0.2431   3.354  0.00080 ***
## ---
## Signif. codes:  0 '***' 0.001 '**' 0.01 '*' 0.05 '.' 0.1 ' ' 1
## 
## Residual standard error: 3.471 on 9704 degrees of freedom
## Multiple R-squared:  0.7861, Adjusted R-squared:  0.7857 
## F-statistic:  2097 on 17 and 9704 DF,  p-value: < 2.2e-16
\end{verbatim}

\begin{enumerate}
\def\labelenumi{\arabic{enumi}.}
\setcounter{enumi}{14}
\tightlist
\item
  Is there an interaction between depth and lakename? How much variance
  in the temperature observations does this explain?
\end{enumerate}

\begin{quote}
ANSWER: The low p-value tells us that there is a significant interaction
between depth and lakename, and the we can see the specific significant
interactions in the summary. The only interaction that is marginally
significant is the interaction of depth and Paul Lake at .0966. 78.57\%
of the variance in temperature observations can be explained by the
interaction between depth and lakename.
\end{quote}

\begin{enumerate}
\def\labelenumi{\arabic{enumi}.}
\setcounter{enumi}{15}
\tightlist
\item
  Create a graph that depicts temperature by depth, with a separate
  color for each lake. Add a geom\_smooth (method = ``lm'', se = FALSE)
  for each lake. Make your points 50 \% transparent. Adjust your y axis
  limits to go from 0 to 35 degrees. Clean up your graph to make it
  pretty.
\end{enumerate}

\begin{Shaded}
\begin{Highlighting}[]
\CommentTok{#16 Plot with temperature by depth for each lake}
\NormalTok{tempbydepth.plot <-}\StringTok{ }\KeywordTok{ggplot}\NormalTok{(NTL.lake.chem.phys.processed, }
  \KeywordTok{aes}\NormalTok{(}\DataTypeTok{x =}\NormalTok{ depth, }\DataTypeTok{y =}\NormalTok{ temperature_C, }\DataTypeTok{color =}\NormalTok{ lakename)) }\OperatorTok{+}\StringTok{ }
\StringTok{  }\KeywordTok{geom_point}\NormalTok{(}\DataTypeTok{alpha =}\NormalTok{ .}\DecValTok{5}\NormalTok{) }\OperatorTok{+}\StringTok{ }
\StringTok{  }\KeywordTok{geom_smooth}\NormalTok{(}\DataTypeTok{method =} \StringTok{"lm"}\NormalTok{, }\DataTypeTok{se =} \OtherTok{FALSE}\NormalTok{) }\OperatorTok{+}
\StringTok{  }\KeywordTok{xlim}\NormalTok{(}\DecValTok{0}\NormalTok{, }\DecValTok{10}\NormalTok{) }\OperatorTok{+}
\StringTok{  }\KeywordTok{ylim}\NormalTok{(}\DecValTok{0}\NormalTok{, }\DecValTok{35}\NormalTok{) }\OperatorTok{+}
\StringTok{  }\KeywordTok{scale_color_brewer}\NormalTok{(}\DataTypeTok{palette =} \StringTok{"Paired"}\NormalTok{) }\OperatorTok{+}
\StringTok{  }\KeywordTok{labs}\NormalTok{(}\DataTypeTok{x =} \StringTok{"Depth"}\NormalTok{, }\DataTypeTok{y =} \StringTok{"Temperature (Celsius)"}\NormalTok{, }
              \DataTypeTok{fill =} \StringTok{"lakename"}\NormalTok{, }\DataTypeTok{color =} \StringTok{"Lake Name"}\NormalTok{) }\OperatorTok{+}
\StringTok{  }\KeywordTok{theme}\NormalTok{(}\DataTypeTok{legend.position =} \StringTok{"right"}\NormalTok{)}
\KeywordTok{print}\NormalTok{(tempbydepth.plot)}
\end{Highlighting}
\end{Shaded}

\begin{verbatim}
## Warning: Removed 683 rows containing non-finite values (stat_smooth).
\end{verbatim}

\begin{verbatim}
## Warning: Removed 683 rows containing missing values (geom_point).
\end{verbatim}

\begin{verbatim}
## Warning: Removed 21 rows containing missing values (geom_smooth).
\end{verbatim}

\begin{center}\includegraphics{A06_GLMs_files/figure-latex/unnamed-chunk-6-1} \end{center}


\end{document}
